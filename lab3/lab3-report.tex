\documentclass[10pt]{article}
\usepackage{graphicx}
\usepackage{subcaption}
\usepackage[T1]{fontenc}
\usepackage{amsmath}
\usepackage{lipsum}
\usepackage{amsfonts}
\usepackage{hyperref}
\usepackage[utf8]{inputenc}
\usepackage[letterpaper,margin=1in]{geometry}
\usepackage[parfill]{parskip}
\usepackage[europeanresistors, american]{circuitikz}
\usetikzlibrary{arrows,shapes,calc,positioning}
\usetikzlibrary{shapes}
\usetikzlibrary{plotmarks}
\usepackage{tikz}
\usepackage{pgfplots}
\usepackage{xparse}
\usetikzlibrary{decorations.markings,positioning}
\definecolor{whitesmoke}{rgb}{0.90, 0.90, 0.90}
\definecolor{lightgray}{rgb}{0.73, 0.73, 0.73}% overrides default?
\definecolor{dimgray}{rgb}{0.51, 0.51, 0.51}
\pgfplotsset{compat=newest}
\usepgfplotslibrary{units}
\newcommand{\oscope}[2] % #1 = name , #2 = rotation angle
{
    \draw[thick,rotate=#2] (#1) circle (12pt)
    (#1) ++(-0.35,-0.1) -- ++(0.3,0.3) --++(0,-0.3)-- ++(0.3,0.3) --++(0,-0.3);
}
\def\therefore{\boldsymbol{\text{ }
\leavevmode
\lower0.4ex\hbox{$\cdot$}
\kern-.5em\raise0.7ex\hbox{$\cdot$}
\kern-0.55em\lower0.4ex\hbox{$\cdot$}
\thinspace\text{ }}}

\vspace{-8ex}
\date{}
\begin{document}

\title{\textbf{\Large{\textsc{ECE320:} Fields and Waves}} \\ \Large{Lab 3 Report: Design of a Double Stub Matching Network} \\ \textbf{\small{PRA106}}\vspace{-0.3cm}}
\author{Alp Tarım, Pranshu Malik \\ \footnotesize{1003860128}, \footnotesize{1004138916}}

\maketitle

\section{Introduction}

This laboratory session was focused on investigating the voltage standing wave (VSW) pattern along a microstrip transmission line, 
as well as its depedance on the load impedance. Figure 1 shows the schematic for a double-stub tuner and its equivalent circuit. 

\begin{figure}[h]
  \centering
  \begin{circuitikz} 
    \draw
    (0,2) to [sqV, l_=$\tilde V_g$] (0, 0) -- (2,0)
    to [tline, l_=Transmission Line, o-o] (7,0)
    to [R, l=$Z_L$] (7, 2)
    to [tline, l=${Z_0, \beta, L}$, o-o] (2,2)
    to [R, l_=$Z_g$, i<^=$\tilde I$] (0, 2);
    
    \draw (2,2.35) node{$\tilde V$} (2, 2);
    \draw (2,0) -- (2.5,-0.15) to[sV, l_=\footnotesize{CH1}, color=white, name=CH1] (2.5,1.75) -- (2, 2);
    \oscope{CH1}{0}
    \draw (7,0) -- (8,-0.15) to[sV, l_=\footnotesize{CH2}, color=white, name=CH2] (8,1.75) -- (7, 2);
    \oscope{CH2}{0}
    \draw (0,0) to[short, *-*] node[ground]{} (0,0);
    \draw [dotted] (2,-0.35) -- (2,0.35) (7,-0.35) -- (7,0.35)
    (2.1, -0.5) node{$z=-L$} (2, -0.35) (7, -0.5) node{$z=0$} (7, -0.35);
  \end{circuitikz}
  \caption{A double-stub matching network}
\end{figure}

\section{Measurement of the Unknown Load Impedance}

\begin{table}[h]
  \[
      \begin{array}{c|c}
          \textbf{Parameter} & \textbf{Value} \\ \hline
          L_1 & 12.15 \text{ cm}\\
          L_2 & 48.697 \text{ cm}\\
          L_1' & 16.88 \text{ cm}\\
          L_2' & 16.69 \text{ cm}
      \end{array}
  \]
  \caption{Theoretically calculated stub length pairs}
\end{table}

\section{Smith Charts and the Graphical Matching Process}
\section{Designing a Double-stub Matching Network}
\section{Experimental Determination }
\section{Bandwidth Calculations}
\section{Notes}

All images taken during the lab were post-processed in a batch using a custom script
that bit-wise inverts the pixels and binarizes the resulting image based on a custom threshold.
No adjustments or modifications were made to the readings, for which the measurements on the VNA
are also shown alongside the waveforms. All scripts and related work can be found at 
\href{https://github.com/pranshumalik14/ece320-labs}{\texttt{github.com/pranshumalik14/ece320-labs}}.

\end{document}